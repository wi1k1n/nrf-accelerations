\chapter{Method}
\label{chap:method}

In this chapter I describe the proposed solution


\section{NSVF with light interaction}

NSVF proposed a good technique of using Sparse Voxel Trees in order to increase the rendering speed. However, the scene is still lacking the light interaction.

One can achieve this by also passing light directions along with distance to the light source into the network, in order to make it distinguish different surface properties for different view and light directions.





\section{Tangent frame for light/view directions}

Since the appearance effects happen mostly in a local coordinate frame, the usage of the global direction vectors implies on the model to learn global to local coordinate system transformation. Although this is generally achievable, the overall complexity of the task can be too overwhelming for the model and some kind of correlations might affect the model. In order to increase the training performance this transformation can be done deterministically and the view and light directions in local coordinate frame are to be passed to the input of the model.

The usage of the cartesian vectors is not very effective for reflectance representations. The half diff vectors (rusinkiewicz parametrization) can be used instead of positionally encoded l and v.




\section{Light rays sampling}

NRF used the colocated light sources. In this case the light rays sampling was coinciding with the view ray sampling. The 








% This is some test area for new mathematical helper macros to nicely visualize mathematical formulas.

% \section{Numbers}
% \begin{align}
%     \mathbb{C}
%     \qquad
%     \mathbb{R}
%     \qquad
%     \mathbb{Q}
%     \qquad
%     \mathbb{Z}
%     \qquad
%     \mathbb{N}
% \end{align}

% \section{Numbers with physical units}
% \begin{align}
%     \SI{1.23}{\meter\per\second}
% \end{align}
% \begin{align}
%     \si{\meter\per\second}
% \end{align}
% \begin{align}
%     \SI{1.23\pm0.45}{\meter\per\second}
% \end{align}
% \begin{align}
%     \SI{3e8}{\meter\per\second}
% \end{align}
% \begin{align}
%     \SI{32}{\giga\byte} = \SI{32e9}{\byte}
% \end{align}
% \begin{align}
%     \SI{32}{\gibi\byte} = \SI[exponent-base=2]{32e30}{\byte}
% \end{align}

% \section{Norm, Dot, Abs, Interval}
% \begin{align}
%     \pi = \const
% \end{align}
% \begin{align}
%     1 \in \interval{0}{2}
% \end{align}
% \begin{align}
%     1 \in \order{n}
% \end{align}
% \begin{align}
%     \evalat{ \frac{\partial f}{\partial x} }{ x = 0 }
% \end{align}
% \begin{align}
%     \norm{p} \qquad \norm{\frac{p}{2}}
% \end{align}
% \begin{align}
%     \abs{p} \qquad \abs{\frac{p}{2}}
% \end{align}
% \begin{align}
%     \dotproduct{p}{q} \qquad \dotproduct{\frac{p}{2}}{q}
% \end{align}
% \begin{align}
%     \crossproduct{p}{q} \qquad \crossproduct{\frac{p}{2}}{q}
% \end{align}

% \section{Vector, Matrix}
% \begin{align}
%     \vec{p} \qquad \vecarrow{p}
% \end{align}
% \begin{align}
%     \vec{p}^{\transposed}
% \end{align}
% \begin{align}
%     \gradient{\vec{p}}
% \end{align}
% \begin{align}
%     \divergence{\mat{A}}
% \end{align}
% \begin{align}
%     \laplacian{\mat{A}}
% \end{align}
% \begin{align}
%     \mat{A}
% \end{align}
% \begin{align}
%     \set{K} , K
%     \qquad
%     \set{N} , N
% \end{align}
% \begin{align}
%     \neighborhood{\vec{p}} = \left\{ \vec{q} \mid \norm{\vec{p} - \vec{q}} < \epsilon \right\}
% \end{align}

% \section{Set operations}
% \begin{align}
%     A \intersect B
% \end{align}
% \begin{align}
%     A \union B
% \end{align}
% \begin{align}
%     A \difference B
% \end{align}

% \section{Derivative, Integral, Sum, Probability}
% \begin{align}
%     \int_H x \, dx
% \end{align}
% \begin{align}
%     \sum_H x
% \end{align}
% \begin{align}
%     \probability{x}
% \end{align}
% \begin{align}
%     \probabilitygiven{x}{y}
% \end{align}
% \begin{align}
%     \expectation{x}
% \end{align}
% \begin{align}
%     \deviation{x}
% \end{align}
% \begin{align}
%     \variance{x}
% \end{align}


% \section{Lemma, Theorem, Corollary}
% \begin{lemma}
%     This is a lemma.
% \end{lemma}
% \begin{proof}
%     Proof of lemma.
% \end{proof}

% \begin{theorem}
%     This is a theorem.
% \end{theorem}
% \begin{proof}
%     Proof of theorem.
% \end{proof}

% \begin{corollary}
%     This is a corollary.
% \end{corollary}
% \begin{proof}
%     Proof of corollary.
% \end{proof}


